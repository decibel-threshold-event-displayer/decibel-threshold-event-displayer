\section{Introduction}\label{sec:introduction}

\subsection{Initial Situation}\label{subsec:initial-situation}
According to the~\href{https://www.bafu.admin.ch//}{Federal Office for the Environment} (DE: BAFU) one in seven people in Switzerland is affected by noise pollution~\cite{foen_noise_pollution}.
The pollution comes primarily from road traffic, followed by railways and then air traffic.
In addition to these noise sources, construction sites, nightclubs and public facilities also produce noise.
Because of this, Switzerland has set up upper noise limits that must be respected.
However, it is of course the case that these limits are not always adhered to.
Affected people must then either accept this or take action against it.
For the latter, they must gather evidence to prove their noise disturbance to the police and the courts.
This evidence then comes from an audio recording which the affected person made them self.

\subsection{Project Goal}\label{subsec:project-goal}
To help the people affected by noise pollution, we want to create an application which processes a given sound file (.wav)
and analyzes it.
It should detect when a specified threshold has been exceeded and then summarizes the result in a PDF document.
The document should then contain all necessary information for filing a complaint.
As our application will have a wide range of end users, two of our main design goals are to make it as user-friendly as possible
and to make it platform independent, so it can be used with any PC operating system and ideally mobile device.

\subsection{Problems with Audio Files}\label{subsec:problems-with-audio-files}
A wave file (.wav) contains samples of the recorded audio, where each sample represents the amplitude at a given moment.
Those amplitude values are relative to each other and not absolute.
It is therefore impossible to determine the actual dB(A) (loudness relative to the human ear) someone would perceive without any further information~\cite{adobe_community_how_to_know_the_real_world_db_level_of_a_file,stackoverflow_how_can_i_calculate_audio_db_level,stackexchange_exteracting_sound_pressure_from_wav_file}.
Because of this, we require the user to also give information about the minimal and maximal dB(A) measured in the given audio file.
To do this, we recommend to use a smartphone app like \href{https://apps.apple.com/ch/app/dezibel-x-dba-l%C3%A4rm-messger%C3%A4t/id448155923}{DecibelX for IOS},
which allows the user to record the audio and also conveniently inspect the minimal and maximal dB measured in that recording.
With those two values, we can then map the relative values from the .wav file to its db(A) values.