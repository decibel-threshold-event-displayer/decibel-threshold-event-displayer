\section{Specification}
The following chapter describes the system specification. The specification is derived from the requirements in the project description
as well as the discussions with the stakeholder. Assumptions and constraints are described in the following sections and were validated by
the defined product owner and the stakeholders if necessary.

\subsection{System Delimitation}
The system delimination is split into the static system environment (\ref{subsubsec:system_environment})
and the dynamic process environment (\ref{subsubsec:process_environment}).

\subsubsection{System Environment}
\label{subsubsec:system_environment}
System:
\begin{itemize}
    \item Frontend (User interaction)
    \item LaTeX and pgfplots
\end{itemize}
System context:
\begin{itemize}
    \item Stakeholder (tutor)
    \item User
    \item Lärmliga
    \item (Accurate conversion to absolute decibel)
\end{itemize}
Out of scope:
\begin{itemize}
    \item Integration in legal complaints
    \item (Accurate conversion to absolute decibel)
\end{itemize}

\subsubsection{Process Environment}
\label{subsubsec:process_environment}
\begin{itemize}
    \item Selecting a file
    \item WAV file analysis (validation, parsing, conversion to absolute db values, threshold filtering)
    \item Plotting (LaTeX / pgfplots) and PDF generation
\end{itemize}

\subsection{Requirements}
In the following section the requirements are detailed. Also the project boundaries and pre-conditions are described.

\subsubsection{Functional requirements}
The project team identified the following functional requirements:
\begin{table}[H]
    \centering
    \begin{tabularx}{\textwidth}{|c|X|c|c|c|}
        \hline
        \textbf{ID} & \textbf{Requirement} & \textbf{Priority} \\
        \hline
        R1 & Allow users to upload a .wav audio file. & MUST \\
        \hline
        R2 & Validate the uploaded file to ensure it is in .wav format, providing an error message if it is not. & MUST \\
        \hline
        R3 & Analyze the uploaded .wav file and calculate the noise level in decibels (dB). & MUST \\
        \hline
        R4 & Allow users to download the plotted noise data as an image or PDF. & MUST \\
        \hline
        R5 & Allow users to input metadata for the audio file, such as location, date, and time. & MUST \\
        \hline
        R6 & Plot the noise level data over time, with the x-axis representing time and the y-axis representing noise level (in dB). & MUST \\
        \hline
        R7 & Generate a PDF report including the plotted noise data and user input metadata. & MUST \\
        \hline
        R8 & Provide clear feedback and error messages. & SHOULD \\
        \hline
        R9 & Intuitive and responsive UI for selecting files, configuring options, and viewing results. & SHOULD \\
        \hline
        R10 & Allow the user to configure custom thresholds for noise levels. & COULD \\
        \hline
        R11 & Allow the user to change the language of the application. & COULD \\
        \hline
    \end{tabularx}
    \caption{Functional Requirements}
    \label{table:functional_requirements}
\end{table}

\subsubsection{Boundary and Pre-Conditions}
Lorem ipsum

\subsection{Usability}
Lorem ipsum

\subsubsection{Personas}
Lorem ipsum

\subsubsection{Storyboard}
Lorem ipsum

\subsubsection{UX-Prototyping}
Lorem ipsum