\section{Scrum Adaptations}
\subsection{Product Owner}
As described in \autoref{sec:scrum_roles}, the Product Owner role has gone from our tutor, Dr.\ Simon Kramer, to Dominic Gernert. Although not specifically mentioned in the Scrum Guide\cite{scrum_guide}, in the experience
of the project team and an article on applied frameworks\cite{applied_frameworks_po}, the Product Owner is usually someone with a business background and not a developer. This makes sense, because they need to be able to make decisions about the product that only a business
expert could make. Since this is not the real world but a school project, the project team had to reduce the coordination effort. Thus, the team decided to make Dominic Gernert the Product Owner. Since he is
not a business expert for our project, decisions are still made as a collective. If however, there is a disagreement on a decision, the Product Owner's suggestion is followed. For bigger decisions,
the Stakeholder Dr.\ Simon Kramer is involved.
\subsection{Daily Scrum}
As suggested by Mr.\ Frank Helbling in the Scrum presentation\cite{helbling_scrum3}, the project team has decided to confine their Daily Scrum to 15 minutes. They are scheduled weekly on Wednesday at 18:15.
The Scrum meetings that took place until the completion date of this document took about 30 minutes, however. The likely reason for this is that the project team meets only weekly,
but each person actually works on the project at multiple days of the week. This leads to more information, questions, and impediments being generated and having to be discussed during the Daily Scrum.
The project team is still trying to keep their Daily Scrum short (less than 20min.) in the future.
\subsection{Release Plan}
The project team has decided against implementing a Release Plan. Mainly because it was never explicitly demanded by the tutor, Dr.\ Simon Kramer. The other reason for forgoing the Release Plan is that
there is no real customer awaiting product updates and Dr.\ Kramer is capable of running the application himself should he want to do so. Product demonstrations by the team are not on a fixed scheduled and
are done if necessary or requested by the stakeholder.
\subsection{Retro}
Instead of following the Keep-Try-Drop model for the Sprint Retrospective, the project team decided to focus on successes, problems, and improvements. A reason for that is the
focus on the individual by the team due to the low number of members. This focus on successes and problems provides a way for each team member to explain their frustrations and discuss possible improvements.
This adaptation is comparatively small and mostly one of wording. The project team feared however, that using the Keep-Try-Drop model, they would have to justify using certain tools when running into minor
problems. Instead, the team hopes to focus more on the product than the tool chains.