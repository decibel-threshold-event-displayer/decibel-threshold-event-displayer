\section{Retrospective}\label{sec:retrorespective}

\subsection{Retrospective I: Scrum roles and stakeholders}\label{subsec:retrospective-scrum-roles-and-stakeholder}
% Scrum Roles und Stakeholder: Erfahrungen und Fazite zu Scrum Roles (Product Owner, Scrum Master, Developer) und weiteren Stakeholder.
% Scrum roles and stakeholders: Experiences and conclusions on Scrum Roles (Product Owner, Scrum Master, Developer) and other stakeholders.

\begin{itemize}
    \item Product Owner: Dominic has already quite some experience with Scrum and thus made a perfect job from the beginning.
    \item Scrum Master: Even though Lukas held the role for the first time he could fulfill it after a bit of initial support from Dominic.
    \item Developers: Thanks to the clear definition of ready and definition of done, the developers always knew what to do in every task.
    \item Stakeholder: The project team enjoyed working with Dr. Kramer as the expectations where communicated beforehand and clearly.
\end{itemize}

\subsection{Retrospective II: Scrum Events and Artifacts}\label{subsec:retrospective-scrum-events-and-artifacts}
% Scrum Events und Artifacts: Erfahrungen und Fazite zu Scrum Events und Scrum Artifacts.
% Scrum Events and Artifacts: Experiences and conclusions on Scrum Events and Scrum Artifacts.

\begin{itemize}
    \item Product Backlog: The Product Owner did a perfect job with specifying and managing the product backlog, this helped the team to progress without any blockers.
    \item Sprint planning meeting: The planning was always really smooth, as it was clear which the logical next steps are.
    \item Sprint backlog: The Sprint backlogs where always a bit over ambitious, as we almost never achieved all the sprint goals.
    \item Daily Scrum meetings: The `Daily's` where actually held once a week, nevertheless they helped us to resolve issues on the current tasks and kept all team members in sync.
    \item Burndown Charts: As we did not work on the project every day, the burndown charts looked not like a gradual line downwards.
    \item Finished work: As we had a clear definition of done, the finished work was always documented in the projects report.
    \item Sprint Review: The project team really enjoyed the sprint reviews, as it made the progress visible.
    \item Sprint Retrospective: At the start of the project, we decided to use a minimal approach for the retrospective and the project team was really happy with this decision.
\end{itemize}

\subsection{Retrospective III: Tools/Instruments}\label{subsec:retrospective-tools-instruments}
% Tools/Instrumente: Erkenntnisse zu Tool-Einsatz (GitLab/GanttLab, Teams, etc.) und Controlling (Burndown Chart, etc.).
% Tools/Instruments: Findings on tool use (GitLab/GanttLab, teams, etc.) and controlling (burndown chart, etc.).

\begin{itemize}
    \item GitLab: The project team was not always happy with the workflows provided by GitLab, such as the issue boards and the differentiation between group and project.
    \item MS Teams: This tool is an amazing, unified communication channel for working together, but the Linux support could be better.
    \item \href{https://www.microsoft.com/en-us/microsoft-365/outlook/email-and-calendar-software-microsoft-outlook}{MS Outlook}: MS Outlook just works as expected.
    \item LaTeX: Even though we did not have a lot of experience with LaTeX we decided to use it for all documentation and presentation purposes.
          The project team enjoyed working with LaTeX, but for non-academic contexts we would have chosen less complex tools such as MS Word and MS PowerPoint.
    \item \href{https://excalidraw.com/}{Excalidraw}: This easy to use online tool helped us to create a lot of the diagrams and UI-Prototypes.
    \item \href{https://www.drawio.com/}{draw.io}: Another easy to use online tool for creating diagrams where the results look a bit more old-school / professional.
    \item Controlling:
    \begin{itemize}
        \item Burndown Chart: The Burndown chart was a good visual indicator to see how the sprints progress is going, but as we often underestimated tasks it could be quite misleading for an external observer.
        \item Product Backlog: Thanks to the detailed initial product backlog we could always see where we stand in the project.
    \end{itemize}
\end{itemize}