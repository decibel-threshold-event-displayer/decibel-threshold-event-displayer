\section{Project Setup Review}\label{sec:project-setup-review}
% Erfassung der Ausgangssituation, Themen-Analyse, Stakeholder(-Management), Organisation, Installationen
% Identifying the initial situation, topic analysis, stakeholder (management), organisation, installations

\subsection{Identifying the initial situation}\label{subsec:identifying-the-initial-situation}
For identifying the initial situation we analysed~\fullref{subsec:project-description} and used multiple online resources
such as the websites of \href{https://laermliga.ch/}{Lärmliga Schweiz} and \href{https://www.bafu.admin.ch/bafu/en/home/topics/noise/in-brief.html}{Federal Office for the Environment FOEN}.
Based on those information we were able to derive some of the requirements and specifications for our application,
which now provides a convenient way for affected people to document noise pollution.

\subsection{Topic analysis}\label{subsec:topic-analysis}
To solve the issue at hand, one needs knowledge about how audio recoding and measurement works.
We invested the time needed to analyse the topic and gather the know-how needed to process and analyse audio files.
A short summary of the problematic can be found under~\fullref{subsec:problems-with-audio-files}.
This analysis provided the technical basis for achieving the product goal.

\subsection{Stakeholder / Stakeholder Management}\label{subsec:stakeholder-management}
Initially we identified a list of possible stakeholders:

\begin{itemize}
    \item Tutor: Dr. Simon Kramer
    \item People affected by noise pollution
    \item Noise producers (Construction Sites, Night Clubs, Highways)
    \item Lärmliga Schweiz
    \item Federal Office for the Environment FOEN
\end{itemize}

For the context of this module we decided to only consider our tutor as the sole stakeholder,
because otherwise the stakeholder management would cost much more time without much benefit.
This way the stakeholder management turned out to be straight forward, as Dr. Kramer provided his expectations and the
scope of the project beforehand, only impediments and key decisions had to be discussed together.

\subsection{Organisation}\label{subsec:organisation}
The project organisation and how the project team implemented scrum is documented under the following three chapters:

\begin{itemize}
    \item \fullref{sec:scrum_roles}: The distribution of the Scrum Roles has been proven practicable, even though some of the project members didn't have any experience in their respective roles
    \item \fullref{sec:scrum-adaptations}: The project team was really happy about the way we adapted Scrum in this project as it was really product focused and did not introduce too much overhead
    \item \fullref{sec:requirements}: We found the definition of ready and the definition of done were especially helpful, as it made sure the tickets had a minimum quality, a clear achieveable goal and are well understood by the team.
\end{itemize}

Further the project team mainly worked together remotely which reduced the unnecessary over head of commuting.

\subsection{Installations}\label{subsec:installations}
The project team used the following tools for carrying out project 1:
\begin{itemize}
    \item Project Management: \href{https://about.gitlab.com/}{GitLab}
    \item Version Control (Code and Documentation): \href{https://about.gitlab.com/}{GitLab}
    \item Documentation and Presentation: \href{https://www.latex-project.org/}{LaTeX}
    \item E-Mail: \href{https://www.microsoft.com/en-us/microsoft-365/outlook/email-and-calendar-software-microsoft-outlook}{MS Outlook}
    \item Team communication (Chat and Video Calls): \href{https://www.microsoft.com/en-us/microsoft-teams/group-chat-software}{MS Teams}
    \item Visualizations: \href{https://excalidraw.com/}{Excalidraw}
    \item Diagrams: \href{https://www.drawio.com/}{draw.io}
\end{itemize}
