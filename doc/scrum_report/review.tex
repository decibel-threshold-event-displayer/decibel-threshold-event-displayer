\section{Review}\label{sec:review}
% Produktziel erreicht? Sprintziele? Abgrenzung? Lieferobjekte erstellt? Product Backlog / Sprint Backlogs?
% Product target achieved? Sprint targets? Delimitation? Deliverables created? Product backlog / sprint backlogs?

\subsection{Product goals}\label{subsec:product-goals-review}
The following list summarizes the~\fullref{subsec:product-goals}:

\begin{enumerate}
    \item Analyze Audio File
    \item Summarize findings in a PDF
    \item Easy to use
\end{enumerate}

From our point of view, the first two product goals are clearly achieved, as we built a working application
which can parse and analyze audio files and then summarizes the findings in a PDF document.
The third goal of 'easy to use' is not as straightforward to measure, but we argue that this is achieved by considering the following
list of best practices:

\begin{enumerate}
    \item Cross-platform and no download/installation required: Achieved by building a public web application
    \item No login or personal information required
    \item Clarity, consistency, responsiveness and familiar patterns: Achieved by using~\href{https://getbootstrap.com/2.0.2/}{Bootstrap, from Twitter}
    \item Help and documentation: Achieved with integrating tooltips where needed and show human-readable error messages
    \item Performance and fast load times: Achieved by realizing the application as a minimal SPA and preloading all necessary LaTeX resources in the background while the user is filling out the form
\end{enumerate}

\subsection{Sprint goals}\label{subsec:sprint-goals-review}
The sprint goals were defined in their respective sprints and are documented under~\fullref{sec:sprint-goals}.
We often did not fully achieve the defined sprint goals in the respective sprint
because we underestimated the amount of work going into solving a ticket, especially the effort for the documentation.
So even though on paper it looks like we did not reach our goals, we made good progress in each sprint and finalized the
goal in the followup sprint. Kep in mind that the sprint goals were achieved implicitly and are not mentioned again
in the followup sprint.
In other words, we still achieved all sprint goals, just not in the sprint they were defined.

\subsection{Product delimitation}\label{subsec:product-delimination-review}
The product delimitation is split into~\fullref{subsubsec:system_environment} and~\fullref{subsubsec:process_environment}.
With the help of these two delimitations, we were able to clearly differentiate what is relevant for the product and what is out of scope.

\subsection{Deliverables}\label{subsec:deliverables-review}
All deliverables were specified in their respective tickets and according to our definition of done they must be documented in the project report.
We do not list all deliverables here, as they can be viewed as part of the projects report:~\href{https://gitlab.ti.bfh.ch/decibel-threshold-event-displayer/decibel-threshold-event-displayer/-/raw/main/doc/report/report.pdf}{Project Report}.
Our main deliverable is the final application, which is available online under the following link:~\href{https://decibel-threshold-event-displayer.github.io/#/}{dB threshold event displayer}.

\subsection{Product backlog / Sprint backlog}\label{subsec:backlogs-review}
The product and sprint backlogs are documented under~\fullref{subsec:product_backlog} and ~\fullref{subsec:sprint-backlog} respectively.
The project team was happy to use them as defined. Even though there are still tickets left in the product backlog,
we have no open tickets with the priority medium or high. This means we achieved a minimal viable product within the module's time frame.
