\section{Sprint Goals}\label{sec:sprint-goals}
The sprint goals are defined by the project team at the Sprint Planning right after the conclusion of the previous sprint.
They are formulated to be compliant with the S.M.A.R.T principles.
The project team decides the sprint goals based on the backlog
prioritization and the issues assigned to the next sprint.
The sprint goals are defined and reviewed in a Markdown file in the repository (\href{https://gitlab.ti.bfh.ch/decibel-threshold-event-displayer/decibel-threshold-event-displayer/-/blob/main/doc/scrum/sprints.md}{Gitlab}).

\subsection{Sprint 1}\label{subsec:sprint-1}
For the first sprint, the project team decided to focus on research and prototyping.
This can be understood as a feasibility study.
Due to the team's relative inexperience with LaTeX and due to the project scope, they felt that having a working prototype had to be
made before a design decision could realistically be made. \\
\begin{table}[H]
    \centering
    \begin{tabularx}{\textwidth}{X c}
        \toprule
        \textbf{Goal}                                                                                    & \textbf{Reached} \\
        \midrule
        Prototypes with two different technologies are implemented and their pros and cons are evaluated & Yes              \\
        \midrule
        Tech stack for the project has been chosen                                                       & Yes              \\
        \midrule
        Licence is defined                                                                               & No               \\
        \midrule
        Git repository and documentation skeleton are created                                            & No               \\
        \bottomrule
    \end{tabularx}
    \caption{Sprint goals of sprint 1}\label{tab:sprint_goals1}
\end{table}
In sprint 1, the project team managed to complete the first two goals (prototyping and tech stack evaluation).
However, they were unable to evaluate the licensing because the decision about which tech stack should be used was made only at the end of the sprint.
The project team underestimated the issue weights, leading to goals that were not reached.

\subsection{Sprint 2}\label{subsec:sprint-2}
For the second sprint we already had to set the focus on the intermediate presentation, further we had to finalize the
groundwork such as specifying Requirements, System delimitation, UX-Prototypes and how to process audio files.
Although the project team defined more goals for the second sprint than the first, it is important to note that the total weight of tasks assigned to this
sprint is equal to the first sprint and estimates are more conservative.
\begin{table}[H]
    \centering
    \begin{tabularx}{\textwidth}{X c}
        \toprule
        \textbf{Goal}                                         & \textbf{Reached} \\
        \midrule
        Intermediate presentation is prepared and presented   & Yes              \\
        \midrule
        Requirements are specified                            & Yes              \\
        \midrule
        UX-Prototype is defined                               & Yes              \\
        \midrule
        System delimitation is specified                      & Yes              \\
        \midrule
        Decibel values can be calculated                      & Yes              \\
        \midrule
        Licence is defined                                    & Yes              \\
        \midrule
        Git repository and documentation skeleton where created & Yes            \\
        \bottomrule
    \end{tabularx}
    \caption{Sprint Goals of Sprint 2}\label{tab:sprint_goals2}
\end{table}
In this sprint the whole team made a big effort to achieve all the goals and finished the most important groundwork in
the project, and we have now a clear way forward to build the product.
Even though the criteria for the intermediate presentation were available on a short notice and creating a presentation
in LaTeX turned out to be a bit challenging, the presentation was prepared in time and went well.

\subsection{Sprint 3}\label{subsec:sprint-3}
Based on the important groundwork in the last sprint, we could now start with the implementation of the application
and the deployment and distribution setup.
The amount of sprint goals is similar to the previous sprint, but we went with a bit less story points as we will also
have to absolve the BFH project week 3.
\begin{table}[H]
    \centering
    \begin{tabularx}{\textwidth}{X c}
        \toprule
        \textbf{Goal}                                         & \textbf{Reached} \\
        \midrule
        Write documentation for interface                     & Yes              \\
        \midrule
        Read and parse *.wav files correctly                  & Yes              \\
        \midrule
        Filter audio data correctly                           & No               \\
        \midrule
        Enable repository mirroring for distribution          & Yes              \\
        \midrule
        Implement MVP frontend application                    & No               \\
        \bottomrule
    \end{tabularx}
    \caption{Sprint Goals of Sprint 3}\label{tab:sprint_goals3}
\end{table}
Because the BFH special week 3 took our focus entirely (we were in the same team),
we could not complete all tasks in this sprint.
Nonetheless, we were able to finalize the first implementation and deployment tasks.
Getting started again with JavaScript was a bit challenging,
but with the realization of a rudimentary javascript test environment we could set another important cornerstone for
a well working application.

\subsection{Sprint 4}\label{subsec:sprint-4}
The focus off this sprint is to progress on the implementation of the application.
The project team only defined 3 goals, because they already contain more than enough story points.
After this sprint we should be able to test each system component individually.
\begin{table}[H]
    \centering
    \begin{tabularx}{\textwidth}{X c}
        \toprule
        \textbf{Goal}                                         & \textbf{Reached} \\
        \midrule
        Filter audio data correctly                           & No               \\
        \midrule
        Implement MVP frontend application                    & Yes              \\
        \midrule
        Improve javascript test environment                   & No               \\
        \bottomrule
    \end{tabularx}
    \caption{Sprint Goals of Sprint 4}\label{tab:sprint_goals4}
\end{table}
We only completed one of the three sprint goals, because achieving the definition of done, especially completing the documentation,
for each task often took more time than we initially estimated.
However, we made massive progress and could implement and test all the system components individually.

\subsection{Sprint 5}\label{subsec:sprint-5}
In sprint 5 the project team will be able to connect all the system components and test the application for the first time as a whole.
\begin{table}[H]
    \centering
    \begin{tabularx}{\textwidth}{X c}
        \toprule
        \textbf{Goal}                                         & \textbf{Reached} \\
        \midrule
        Finalize and document `Filter audio data correctly`   & No               \\
        \midrule
        Document `Improve javascript test environment`        & Yes              \\
        \midrule
        Create LaTeX template and fill in placeholders        & No               \\
        \midrule
        Render LaTeX to PDF                                   & No               \\
        \bottomrule
    \end{tabularx}
    \caption{Sprint Goals of Sprint 5}\label{tab:sprint_goals5}
\end{table}
On paper we achieved only one of four sprint goals as in the end the project team has not enough time for completing
the documentation part of the tickets.
As a consequence, we decided to add more weight to the tasks, even if we would estimate them lower.
On the other hand we made great overall progress and could generate our first pdf plot via the web gui from an uploaded
*.wav file, which means we have a first version of the application where all system components are connected.

\subsection{Sprint 6}\label{subsec:sprint-6}
According to our plan this should be the last sprint and there is a lot of finalization,
documentation and cleanup do be done on all different fronts.
We decided to go with 5 sprint goals which is really ambitious, because afterward the project would be more or less done.

\begin{table}[H]
    \centering
    \begin{tabularx}{\textwidth}{X c}
        \toprule
        \textbf{Goal}                                         & \textbf{Reached} \\
        \midrule
        Finalize and document `Filter audio data correctly`   & No               \\
        \midrule
        Finalize LaTeX template and fill in placeholders      & No               \\
        \midrule
        Finalize LaTeX to PDF                                 & No               \\
        \midrule
        Finalize SPA                                          & No               \\
        \midrule
        Define legal requirements                             & Yes               \\
        \bottomrule
    \end{tabularx}
    \caption{Sprint Goals of Sprint 6}\label{tab:sprint_goals6}
\end{table}

As we already knew, the sprint goals where really ambitious and more than half of this sprint is during the holiday season,
so we actually completed only one of the five sprint goals.
We still managed to get a lot of things done and even though we still have open tasks on all fronts,
we are almost finished with the application.

\subsection{Finalization}\label{subsec:sprint-finalization}
The project team decided that it does not make sense to start another sprint just for the sake of it, the leftover
sprint goals and tasks will be done in a finalization round, which can be thought of an extension of sprint 6,
as it was the last sprint anyway.

\begin{table}[H]
    \centering
    \begin{tabularx}{\textwidth}{X c}
        \toprule
        \textbf{Goal}                                         & \textbf{Reached} \\
        \midrule
        Finalize and document `Filter audio data correctly`   & Yes              \\
        \midrule
        Finalize LaTeX template and fill in placeholders      & Yes              \\
        \midrule
        Finalize LaTeX to PDF                                 & Yes              \\
        \midrule
        Finalize SPA                                          & Yes              \\
        \midrule
        Final presentation is prepared and presented          & Yes              \\
        \bottomrule
    \end{tabularx}
    \caption{Sprint Goals of Sprint 6}\label{tab:sprint-finalization}
\end{table}